% -*- LaTeX -*-
% -*- coding: utf-8 -*-
%
% michael a.g. aïvázis <michael.aivazis@para-sim.com>
% (c) 2003-2017 all rights reserved
%

\section{introduction}

%-----------------------------------
\begin{frame}
%
  \frametitle{Introduction}
%
  \vskip -3ex
  \begin{itemize}
%
  \item \pyre\ is a strategy for
    \begin{itemize}
    \item managing code complexity
    \item integrating third party tools and libraries into a coherent whole
    \item empowering the end-user to make critical decisions about the composition of an
      application while minimizing the risk of compromising its integrity
    \end{itemize}
%
    \item \pyre\ extends object oriented ideas
      \begin{itemize}
      \item abstract base classes become {\em protocols}
      \item appropriately decorated classes become {\em components}
      \item design and implement by contract
      \end{itemize}
%
    \item \pyre\ is also a powerful computational environment with rich services
      \begin{itemize}
      \item application configuration
      \item launching and staging in serial, parallel, distributed modes
      \item logging and monitoring
      \item special services for interacting with users via the production of structured
        documents
        \begin{itemize}
        \item think \html\ for web applications, remote UIs
        \end{itemize}
      \item name and filesystem abstractions
      \item powerful lazy evaluation mechanisms
      \item seamless access to database back-ends without the need for direct access using
        embedded \sql\ or similar techniques
      \end{itemize}
%
  \end{itemize}
%
\end{frame}

%%% Local Variables:
%%% mode: latex
%%% TeX-master: "../pyre"
%%% End:

% end of file
