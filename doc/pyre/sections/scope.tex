% -*- LaTeX -*-
% -*- coding: utf-8 -*-
%
% michael a.g. aïvázis <michael.aivazis@para-sim.com>
% (c) 2003-2017 all rights reserved
%

% -----------------------------------
\begin{frame}
%
  \frametitle{Outline}
%
  \begin{itemize}
%
  \item We will write a \pyre\ application that stitches together a digital elevation model
    (\dem) by downloading tiles from the \srtm\ archive
%
    \begin{itemize}
    \item become familiar with components, and their properties and behaviors
    \item describe the configuration process from the end user's point of view
    \item build a simple pyre application that instantiates and exercises a component
    \item gradually transform the app into a capability that is integrated into an open ended
      environment
    \end{itemize}
%
  \item Much of the code in these examples is lifted straight out of the \isce\ 3.0 source
    code.
%
  \item As we go along, we will introduce notation that makes it possible to discuss
    application architecture without the complexity of having to show actual source code. The
    notation is inspired by \uml\cite{uml-2005}, but is significantly simpler.
%
  \end{itemize}
%
\end{frame}

% -----------------------------------
\begin{frame}[t]
  \frametitle{Case study: assembling a digital elevation model}
%
  % \vskip -5ex
  We will write an application that stitches together a \dem\ for some region of interest. For
  this example, we will use the \srtm\ archive. The basic idea is very simple:

%
  \begin{columns}
    \begin{column}{.4\textwidth}
      \only<1>{
        \begin{center}
          \includegraphics[width=.95\textwidth]{dem-california}
        \end{center}
      }
      \only<2>{
        \begin{center}
          \includegraphics[width=.95\textwidth]{dem-region}
        \end{center}
      }
      \only<3>{
        \begin{center}
          \includegraphics[width=.95\textwidth]{dem-bbox}
        \end{center}
      }
      \only<4->{
        \begin{center}
          \includegraphics[width=.95\textwidth]{dem-tiles}
        \end{center}
      }
    \end{column}
    %
    \begin{column}{.5\textwidth}
      \begin{itemize}
      \item<2-> the user picks a few points in a region of interest
      \item<3-> we compute an axis-aligned bounding box
      \item<4-> fill the box with $1\si{\degree} \times 1\si{\degree}$ tiles
      \item<5-> download the \dem\ tiles from the \srtm\ data store
      \item<6-> stitch the \dem\ together
      \end{itemize}
    \end{column}
    %
  \end{columns}

  \only<7->{A key aspect of the process is to identify which decisions are best left to the end
    user, and therefore become configuration options for our app.}

%
\end{frame}

%-----------------------------------
\begin{frame}
%
  \frametitle{Outline of the implementation}
%
%
  \begin{itemize}
%
  \item
%
  \end{itemize}
%
\end{frame}


%%% Local Variables:
%%% mode: latex
%%% TeX-master: "../pyre"
%%% End:

% end of file
